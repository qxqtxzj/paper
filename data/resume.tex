\begin{resume}

  \resumeitem{个人简历}

  1993 年 02 月 02 日出生于 浙江省 海宁市。

  2011 年 09 月考入 北京邮电大学 计算机学院 计算机科学与技术专业。

  2015 年 07 月本科毕业并获得 工学 学士学位。

  2015 年 09 月免试进入 清华大学 计算机科学与技术系攻读 工程硕士 学位至今。

  \researchitem{发表的学术论文} % 发表的和录用的合在一起

  % 1. 已经刊载的学术论文(本人是第一作者,或者导师为第一作者本人是第二作者)
  %\begin{publications}
  %  \item Jie Zhu, Qi Li, Cong Wang, Xingliang Yuan, Qian Wang, Kui. Enabling Generic,Verifiable, and Secure Data Search in Cloud Services. (已被 IEEE Transactions on Parallel and Distributed Systems(TPDS) 录用)
  %\end{publications}

  % 2. 尚未刊载,但已经接到正式录用函的学术论文(本人为第一作者,或者
  %    导师为第一作者本人是第二作者)。
  \begin{publications}[before=\publicationskip,after=\publicationskip]
    \item Jie Zhu, Qi Li, Cong Wang, Xingliang Yuan, Qian Wang, Kui Ren. Enabling Generic,Verifiable, and Secure Data Search in Cloud Services. IEEE Transactions on Parallel and Distributed Systems(TPDS), 2018:1-1, DOI: 10.1109/TPDS.2018.2808283)
  \end{publications}

  % 3. 其他学术论文。可列出除上述两种情况以外的其他学术论文,但必须是
  %    已经刊载或者收到正式录用函的论文。
  %\begin{publications}
  %  \item Wu X M, Yang Y, Cai J, et al. Measurements of ferroelectric MEMS
  %    microphones. Integrated Ferroelectrics, 2005, 69:417-429. (SCI 收录, 检索号
  %    :896KM)
  %  \item 贾泽, 杨轶, 陈兢, 等. 用于压电和电容微麦克风的体硅腐蚀相关研究. 压电与声
  %    光, 2006, 28(1):117-119. (EI 收录, 检索号:06129773469)
  %  \item 伍晓明, 杨轶, 张宁欣, 等. 基于MEMS技术的集成铁电硅微麦克风. 中国集成电路,
  %    2003, 53:59-61.
  %\end{publications}

  \researchitem{研究成果} % 有就写,没有就删除
  \begin{achievements}
    \item 李琦, 朱洁, 王骞. 一种可验证的加密搜索方法: 中国, 201711277295.7. (中国专利申请号)
    \item 李琦,朱洁,陈艳毓,李漓春. 检测软件定义网络(SDN)中的路由环路的系统和方法:中国, 201810437997.5. (中国专利申请号)
    \item Qi Li, Jie Zhu, Yanyu Chen, Lichun Li. System and Method for detecting routing loops in a Software Defined Network (SDN): SG, 10201703959R. (新加坡专利申请号)
  \end{achievements}

\end{resume}
