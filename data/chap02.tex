\chapter{相关研究综述}
\label{cha:related}
现有的CS2论文[5]提出了搜索鉴别符方案,该方案利用了默克尔树(Merkle Hash Tree, MHT)来实现了对数据完整性的验证,然而该方案并没有考虑到服务器返回空结果来规避结果验证的情况。Kurosawa等人提出了一系列可验证加密搜索方案[6][8][12],然而他们的方案要么在搜索效率上接近O(n),不适用于大型数据库,要么只支持静态数据库。除了Kurosawa等人的方案,其他还有许多只支持静态数据库的方案[7][10][13]。虽然Stefanov等人的方案[9]通过消息验证码(Message Authenticated Code,MAC)实现了动态数据库下的可验证搜索,然而他们的方案仍然没有考虑到服务器返回空结果的情况。Bost等人提出的方案是目前为止最完善的[11],但他们的方案在搜索时需要和服务器有两轮通信才能进行验证,无法并行进行验证,并且他们的方案同样也不支持多用户情况下的验证。各个方案的对比情况如表格1所示。
表格 1 现有的可验证加密搜索方案比较
	动态性	多用户	防御重放攻击	防御数据完整性攻击	验证效率
KPR11[5]	√	×	√	×	O(|W|)
KO12[6]	×	×	×	×	O(n)
CG12[7]	×	×	×	√	O(log|W|)
KO13[8]	√	×	√	×	O(n)
KW13[13]	×	√	×	√	O(|W|)
SPS14[9]	√	×	√	×	O(rlog^3 (n))
BFP16[11]	√	×	√	√	O(r)
OK16[12]	×	×	×	√	O(1)
OURS	√	√	√	√	O(log|W|)

2.2 可验证的非对称加密搜索
第一个可验证的非对称加密搜索方案[18]由Zheng等人提出,他们的方案采用了基于属性的关键字(Attribute-based keyword,ABK),但是他们的方案也只适用于数据库静态的情况。基于他们的工作,Liu等人又提出了一个更高效的可验证非对称加密搜索方案[19],然而,由于非对称加密本身的限制,他们的方案必不可少地需要引入一个可信第三方。
2.3 不可验证的多用户加密搜索
	Curtmola等人在2006年即提出了一个基于广播加密的多用户加密搜索方案[2],该方案允许数据拥有者将数据分享给其他用户,并且数据拥有者具有对用户的访问控制权限,可以随时撤销或者新增用户。Jerecki等人随后又提出了一个基于Oblivious Cross Tag的加密搜索方案[21],然而该方案需要数据拥有者和数据用户频繁的交互。
2.4 总结
	综上所述,现有的可验证加密搜索方案都不能满足多用户场景下的安全性保证,并且现有的方案无法完善地解决重放攻击和数据完整性攻击。这需要我们设计合理的机制来防御多用户场景下重放攻击,并且需要我们利用新型的数据结构来完善对数据完整性攻击的防御,尤其是防御服务器返回空结果来规避结果验证的情况。


  The scheme proposed by Kamara et al. [5] constructed a Merkle Tree by storing the key and value pairs in its leaf node, where the key was the encrypted keyword and the value was the encrypted data set that contained the keyword. A data user (aka data owner) only kept the root hash of the Merkle Tree. By reconstructing a root hash through the path of the target nodes, the user can check the integrity of the search result. However, they did not clearly state how to handle the situation that the server maliciously returned an empty result set. Unfortunately, their approach cannot detect such attacks. The key reason is that there is no path that matches a non-existent keyword of the Merkle Tree. One naïve way to address this issue is that they can store the results (including empty sets) of the entire keyword space to the tree, which is not scalable. In addition, this approach is not dynamic friendly. It may require rebuilding the entire tree after data update.
  Kurosawa et al. [8] leveraged Message Authenticated Code (MAC) to ensure data integrity and utilizes RSA accumulator to ensure data freshness during data updates. However, the existing study [12] showed that users in [8] required maintaining all keywords in the datasets to detect cheating of servers. It introduces significant overhead for users to maintain a large set of keywords. Therefore, paper [12] proposed a no-dictionary VSSE scheme, which means users do not need to keep the keywords set as a dictionary. However, their scheme only works in the static setting and does not consider data update scenarios.
  Stefanov et al. [9] also used MAC and timestamp to ensure data integrity and freshness. However, they were still unable verify an empty search result unless the data owner maintained all keywords locally.
  In summary, verification on the three-party model (across multiple users) should allow users to verify an empty result in addition to verification of integrity and freshness of search results. Meanwhile, it is desirable that users do not need to maintain a large set of keywords locally for the verification. We clarified this issue in Section 1.




  \section{Related Work}
  \noindent\textbf{Secure Cloud Storage Scheme.} Verifiable cloud storage services have been extensively studied, e.g., Proof of Data Possession (PDP) \cite{ateniese2007provable, ateniese2008scalable, erway2015dynamic,zhu2012cooperative} and Proof \red{of} Retrievability (POR) \cite{juels2007pors, bowers2009proofs, stefanov2012iris}. \blue{These schemes mainly focused on verifying the integrity of data stored in cloud services and enable restoring data blocks if they are corrupted.} However, they did not ensure \red{the} integrity of search results,  \red{which is the focus of VSSE}.
  Authenticated data structures are used by a set of searching algorithms to verify the integrity of data blocks stored on an untrusted server. Several schemes have been proposed, e.g., Merkle Tree\cite{merkle1987digital}, authenticated hash table\cite{papamanthou2008authenticated}, and \red{authenticated} skip list\cite{pugh1990skip,goodrich2001implementation}. Merkle Tree is the most common structure used to verify data integrity. However, Merkle Tree cannot flexible support data update. %is not a balanced tree when new leaf nodes are added and the searching operations are efficient.
  Moreover, the current verification scheme~\cite{kamara2011cs2} built upon Merkle Tree did not store keyword information in its intermediate node and thus it is not suitable for keyword related searches. An authenticated hash table enabled by the RSA accumulator can be used to verify search results as well. Unfortunate, it has low efficiency in searching and update operations. For example, the search delay of the authenticated hash table is in millisecond level, while that of \name is in microsecond level. Skip list used a multilayer linked list to improve its search efficiency, but the storage overhead is much higher than a tree structure if the keyword information is required in the search path.
   %First, Merkle Tree is not a balanced data structure, and search delay of the authenticated hash table leveraging the RSA accumulator is lower than Merkle Patricia Tree. The search delay of the authenticated hash table is in millisecond level, while that of the authenticated hash table is in microsecond level. Second, the current Merkle Tree scheme proposed by Kamara et al. [5] did not store additional information such as keyword in its intermediate node, so that it cannot enable empty result validation

  \noindent\textbf{Verifiable Searchable Symmetric Encryption.} The CS2 scheme \cite{kamara2011cs2} enabled users to verify the search result by using dynamic search authenticators, but their scheme cannot prevent the attacks that the server maliciously replies an empty result.
  Recently Kurosawa et al.~\cite{kurosawa2012uc,kurosawa2013update,ogataefficient} proposed a few verifiable SSE schemes. However, their schemes either have low search efficiency, or do not support verification upon file update.
  Kurosawa et al. \cite{kurosawa2012uc} required linear search in SSE and did not support dynamic file update. Their extension \cite{kurosawa2013update} achieved dynamic updating but the search complexity was beyond linear time. Recently, Ogata et al. \cite{ogataefficient} presented a generic verifiable scheme. It transforms any SSE scheme to a \textit{no-dictionary} verifiable SSE scheme that did not require the users to keep the keyword set. However, it was still a static approach, which shared the similar shortcoming with~\cite{chai2012verifiable} \cite{cheng2015verifiable}.
  %Also, schemes in~\cite{chai2012verifiable} \cite{cheng2015verifiable} only enabled static SSE.
  Although the verifiable scheme proposed by Stefanov et al.\cite{stefanov2014practical} achieved verifiability by leveraging message authenticated code, it cannot easily detect the data integrity attacks when the server intentionally returned an empty result.
  Bost et al. \cite{bost2016verifiable} presented a generic verifiable dynamic SSE scheme and combined it with the SSE scheme proposed by Stefanov et al. \cite{stefanov2014practical}.
  Yet, their scheme required two round communications for result verification and did not enable verification in the setting of multiple users. Our \name scheme is a generic verifiable SSE scheme that can work with three-party model, which can be \red{more readily} deployed in practice. In particular, it enables search result verification under file update with only one round of communication.


  \noindent\textbf{Verifiable Public Key Encryption with Keyword Search.} The first verifiable attribute-based keyword search (VABKS) was proposed by Zheng et al. \cite{zheng2014vabks}. Similar to the existing SSE schemes above, VABKS only focused on search based on static encrypted data. Liu et.al \cite{liu2014efficient} proposed a more efficient construction based on VABKS, and Sun et.al \cite{sun2015catch} also provided a verifiable scheme VCKS that support conjunctive keyword search. However, due to the limitations of asymmetric encryption schemes, both of the above schemes require an additional trusted authority.


  \noindent\textbf{Multi-User Searchable Encryption.} A few of non-verifiable multi-user schemes have been proposed~\cite{curtmola2011searchable,yang2009multiuser,jarecki2013outsourced,sun2016efficient}. Curtmola et al.~\cite{curtmola2011searchable} first proposed a multi-user SSE scheme based on broadcast encryption. Yang et al.~\cite{yang2009multiuser} proposed a multi-user searchable encryption scheme by leveraging a bilinear map. However, the search delay of the scheme is proportional to the size of the database, which is not suitable for large-scale databases.
  Jarecki et al. \cite{jarecki2013outsourced} designed a multi-user scheme by using Oblivious Cross-Tags (OXT) protocol. However, their scheme required frequent communication between data owners and the users, which incured unnecessary communication overheads. Recently, Sun et al.~\cite{sun2016efficient} proposed a non-interactive multi-user searchable encryption schemes that reduced the interactions between data owner and users. However, the scheme did not support search under data update.
