\chapter{总结与展望}
\label{cha:conclusion}
\section{论文工作总结}
本文主要研究了云存储环境下的可验证加密搜索问题,针对现有可验证加密搜索方案的缺点,本文首先提出了一种普适的、支持用户数据更新的可验证对称加密搜索框架\single,该方案支持单用户情况下的加密搜索结果验证。本文通过严格的方案定义和算法描述对该方案进行了详细的分析和描述,并通过安全性分析和实验结果证明了\single 方案的安全性和高效性。随后,基于\single 方案,本文又提出了一种支持多用户场景下的可验证对称加密搜索框架\multi , 该框架的引入使得数据共享场景下的结果验证成为可能,进一步提升了方案的普适性。

首先,本文介绍了云存储环境下的对称加密搜索、可验证对称加密搜索问题,并分析了几个主流方案运用的算法,比较了不同方案之间的优缺点。本文总结了这些方案普遍存在的问题,对可验证对称加密搜索问题中存在的数据新鲜性攻击和数据完整性攻击进行了正式定义,并对论文需要达到的几个目标进行了说明。

在单用户场景下的可验证对称加密搜索框架\single 中,本文提出了一种基于MPT和IH的验证索引,该验证索引独立于加密搜索方案中的索引,这种解耦使得该方案可以与任意的加密搜索方案结合,为其提供结果验证功能。同时该验证索引还支持用户数据更新,不需要在用户数据产生更新时重构该索引,只需要进行增量更改。在该验证索引的基础上,本文为云服务器和数据搜索用户提供了一套完善的结果验证方案:即由云服务器根据验证索引生成结果证明,由数据搜索用户根据该结果证明对加密搜索结果进行验证。该验证方案确保了云服务器的任何恶意行为都会被用户检测到,确保了加密搜索结果的新鲜性和完整性。特别需要说明的是,该方案能对云服务器返回的空搜索结果进行验证。以往的可验证加密搜索方案中,如果用户搜索的关键字不存在,云服务器不会返回任何证明。这使得一个恶意的云服务器可以对任何关键字声称不存在。本论文提出的验证方案不管关键字存在与否,都需要云服务器提供结果证明,这防止了该情况的产生。通过实验验证,本方案在计算开销和通信开销两个方面引入的开销都很小。与加密搜索方案结合后的实验表明,\single 方案提供的结果验证服务带来的额外开销几乎可以忽略不计。

本文在\single 方案的基础上,又提出了一种多用户场景下的可验证对称加密方案\multi 。相对于单用户场景,多用户场景下的数据持有者和数据搜索用户产生了分离,对数据新鲜性验证来说增加了其复杂性。本方案通过一种基于时间戳链的鉴别符来解决该问题,该鉴别符将验证索引的根哈希与时间戳进行了绑定,使得用户可以通过时间戳验证根哈希的新鲜性。该方案由数据持有者生成该鉴别符并上传给云服务器,数据搜索用户可以通过云服务器以拉取的方式来获得该鉴别符,降低了数据持有者的通信开销。通过实验证明,\multi 方案给数据持有者带来的通信开销很小,并且给数据搜索用户引入的验证延迟也可接受。

此外,本文还通过严格的安全性证明对上述两个方案进行了分析,证明了两个方案的机密性和可验证性,即不泄露用户数据和关键字明文信息,同时还保证了可以验证论文定义的数据新鲜性攻击和数据完整性攻击。

综上所述,本文提供了一种云存储环境下的安全加密搜索方案。该方案可以为各种各样的加密搜索方案提供结果验证功能,支持用户数据更新,支持单用户和多用户等多种场景,并且可以抵抗多种可能的安全性攻击,普适性非常高。

\section{未来工作展望}
本文提出的两种可验证对称加密搜索方案已具有较高的实用性,但未来仍有以下几个可以提升的点:
\begin{itemize}
  \item 第一,本文提出的结果验证方案仅验证加密搜索结果在数量上的完整性以及数据本身的完整性。一些具有不同功能的加密搜索方案,例如支持多关键字查询的加密搜索方案,支持结果排序的加密搜索方案,这些方案对结果验证的功能需求更复杂。针对支持多关键字查询的加密搜索方案,验证方案需要支持对搜索结果数量的交并运算,而支持结果排序的加密搜索方案,验证方案需要支持对搜索结果顺序的验证。这些问题目前尚没有相关研究工作,今后可持续展开研究。
  \item 第二,本文提出的多用户对称加密搜索框架\multi 基于一方写入多方读取的情况。在实际的云存储场景中,还存在多方写入多方读取,多方写入单方读取等情况,针对这些场景的研究目前尚没有。
  \item 第三,本文提出的两种可验证方案,云服务器的计算开销都在次线性级别。而目前已有常数级别计算开销的加密搜索方案,这促使我们寻找常数级别的可验证方案,以在性能上和常数级别的加密搜索方案进行匹配。
\end{itemize}
