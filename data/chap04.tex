\chapter{多用户下的可验证对称加密搜索方案研究}
\label{cha:multi-user}
\section{引言}

\section{系统架构} 三方

\section{方案流程} 三方

\section{算法分析}
\subsection{构建时间戳链}
\subsection{验证时间戳}
\subsection{实例分析} 图片举例说明

\section{安全性分析}

\section{实验结果}
图9和图10评估了系统的总验证开销,包括计算开销和通信开销。我们主要考虑合法用户端的验证计算开销,和数据拥有者端的通信开销。图中的η表示数据拥有者的数据更新频率。
首先,我们考虑合法用户端的验证开销。这部分开销包括用户等待检测点的时间以及执行Check算法和Rebuild算法的时间。由于Rebuild算法的开销可以忽略,因此图9中并未标注该部分开销。如图9所示,蓝色的曲线表示合法用户等待检测点的时间,这里我们假设合法用户的查询时刻在一个更新周期内呈均匀分布,那么用户等待时间的均值就是更新间隔的一半。其余的曲线表示Check算法的执行时间,可以看到,当更新频率在2Hz到1/60Hz不等时,Check算法的执行时间也几乎可以忽略,因此合法用户端的验证开销主要取决于等待检测点所需的时间,即和更新间隔的设置有关。
其次,我们考虑数据拥有者端的通信开销。我们主要考虑由鉴别符带来的开销。由于鉴别符的更新包含两种情况,固定更新导致的鉴别符更新和数据更新导致的鉴别符更新。图10中的曲线充分体现了这两种更新带来的开销。首先,当数据拥有者的更新间隔设置的很小时,通信开销较大,这是因为固定更新太频繁导致的鉴别符的通信开销增大;当数据拥有者的更新间隔设置的很大时,通信开销也较大,这是因为当更新频率一定时,更新间隔越大,在更新间隔内形成的鉴别符的长度会累积增大,最终导致了鉴别符的通信开销增大。
综合以上两种情况,更新间隔可以设置为1000ms至1500ms之间。
