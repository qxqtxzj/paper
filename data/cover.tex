\thusetup{
  %******************************
  % 注意:
  %   1. 配置里面不要出现空行
  %   2. 不需要的配置信息可以删除
  %******************************
  %
  %=====
  % 秘级
  %=====
  secretlevel={秘密},
  secretyear={10},
  %
  %=========
  % 中文信息
  %=========
  ctitle={可验证对称加密搜索问题\\分析与研究},
  cdegree={工程硕士},
  cdepartment={计算机科学与技术系},
  cmajor={计算机技术},
  cauthor={朱洁},
  csupervisor={李琦副研究员},
  %cassosupervisor={陈文光教授}, % 副指导老师
  %ccosupervisor={王聪教授,王骞教授}, % 联合指导老师
  % 日期自动使用当前时间,若需指定按如下方式修改:
  % cdate={超新星纪元},
  %
  % 博士后专有部分
  %cfirstdiscipline={计算机科学与技术},
  %cseconddiscipline={系统结构},
  %postdoctordate={2009年7月——2011年7月},
  %id={编号}, % 可以留空: id={},
  %udc={UDC}, % 可以留空
  %catalognumber={分类号}, % 可以留空
  %
  %=========
  % 英文信息
  %=========
  etitle={Analysis and Research of \\Verifiable Searchable Symmetric Encryption},
  % 这块比较复杂,需要分情况讨论:
  % 1. 学术型硕士
  %    edegree:必须为Master of Arts或Master of Science(注意大小写)
  %             “哲学、文学、历史学、法学、教育学、艺术学门类,公共管理学科
  %              填写Master of Arts,其它填写Master of Science”
  %    emajor:“获得一级学科授权的学科填写一级学科名称,其它填写二级学科名称”
  % 2. 专业型硕士
  %    edegree:“填写专业学位英文名称全称”
  %    emajor:“工程硕士填写工程领域,其它专业学位不填写此项”
  % 3. 学术型博士
  %    edegree:Doctor of Philosophy(注意大小写)
  %    emajor:“获得一级学科授权的学科填写一级学科名称,其它填写二级学科名称”
  % 4. 专业型博士
  %    edegree:“填写专业学位英文名称全称”
  %    emajor:不填写此项
  edegree={Master of Engineering},
  emajor={Computer Technology},
  eauthor={Zhu Jie},
  esupervisor={Professor Li Qi},
  %eassosupervisor={Chen Wenguang},
  % 日期自动生成,若需指定按如下方式修改:
  % edate={December, 2005}
  %
  % 关键词用“英文逗号”分割
}

% 定义中英文摘要和关键字
\begin{cabstract}
  云计算的发展使得用户可以很方便的存储、获取与分享数据。但与此同时,云存储也带来了很多安全问题,例如,数据隐私泄露等。对称加密搜索的提出解决了数据隐私泄露问题,同时也保证了数据的可用性。通过使用对称加密搜索方案,用户可以在上传数据之前,对数据进行加密,同时云服务器可以在用户的加密数据上进行搜索,从而确保数据的隐私性。

  本文的创新点主要有:
  \begin{itemize}
    \item 用例子来解释模板的使用方法;
    \item 用废话来填充无关紧要的部分;
    \item 一边学习摸索一边编写新代码。
  \end{itemize}

\end{cabstract}

% 如果习惯关键字跟在摘要文字后面,可以用直接命令来设置,如下:
\ckeywords{对称加密搜索, 结果验证, 云计算,云存储}

\begin{eabstract}
  Searchable Symmetric Encryption (SSE) has been widely studied in cloud storage, which allows cloud services to directly search over encrypted data. Most SSE schemes only work with honest-but-curious cloud services that do not deviate from the prescribed protocols. However, this assumption does not always hold in practice due to the untrusted nature in storage outsourcing. To alleviate the issue, there have been studies on Verifiable Searchable Symmetric Encryption (VSSE), which functions against malicious cloud services by enabling results verification. But to our best knowledge, existing VSSE schemes exhibit very limited applicability, such as only supporting static database, demanding specific SSE constructions, or only working in the single-user model. In this paper, we propose GSSE, the first generic verifiable SSE scheme in the single-owner multiple-user model, which provides verifiability for any SSE schemes and further supports data updates. To generically support result verification, we first decouple the proof index in GSSE from SSE. We then leverage Merkle Patricia Tree (MPT) and Incremental Hash to build the proof index with data update support. We also develop a timestamp-chain for data freshness maintenance across multiple users. Rigorous analysis and experimental evaluations show that GSSE is secure and introduces small overhead for result verification.
\end{eabstract}

\ekeywords{Searchable Symmetric Encryption, Result Verification, Cloud Computing, Cloud Storage}
