\thusetup{
  %******************************
  % 注意:
  %   1. 配置里面不要出现空行
  %   2. 不需要的配置信息可以删除
  %******************************
  %
  %=====
  % 秘级
  %=====
  secretlevel={秘密},
  secretyear={10},
  %
  %=========
  % 中文信息
  %=========
  ctitle={可验证对称加密搜索算法研究},
  cdegree={工程硕士},
  cdepartment={计算机科学与技术系},
  cmajor={计算机技术},
  cauthor={朱洁},
  csupervisor={李琦副研究员},
  %cassosupervisor={陈文光教授}, % 副指导老师
  %ccosupervisor={王聪教授,王骞教授}, % 联合指导老师
  % 日期自动使用当前时间,若需指定按如下方式修改:
  % cdate={超新星纪元},
  %
  % 博士后专有部分
  %cfirstdiscipline={计算机科学与技术},
  %cseconddiscipline={系统结构},
  %postdoctordate={2009年7月——2011年7月},
  %id={编号}, % 可以留空: id={},
  %udc={UDC}, % 可以留空
  %catalognumber={分类号}, % 可以留空
  %
  %=========
  % 英文信息
  %=========
  etitle={Analysis and Research of \\Verifiable Searchable Symmetric Encryption},
  % 这块比较复杂,需要分情况讨论:
  % 1. 学术型硕士
  %    edegree:必须为Master of Arts或Master of Science(注意大小写)
  %             “哲学、文学、历史学、法学、教育学、艺术学门类,公共管理学科
  %              填写Master of Arts,其它填写Master of Science”
  %    emajor:“获得一级学科授权的学科填写一级学科名称,其它填写二级学科名称”
  % 2. 专业型硕士
  %    edegree:“填写专业学位英文名称全称”
  %    emajor:“工程硕士填写工程领域,其它专业学位不填写此项”
  % 3. 学术型博士
  %    edegree:Doctor of Philosophy(注意大小写)
  %    emajor:“获得一级学科授权的学科填写一级学科名称,其它填写二级学科名称”
  % 4. 专业型博士
  %    edegree:“填写专业学位英文名称全称”
  %    emajor:不填写此项
  edegree={Master of Engineering},
  emajor={Computer Technology},
  eauthor={Zhu Jie},
  esupervisor={Professor Li Qi},
  %eassosupervisor={Chen Wenguang},
  % 日期自动生成,若需指定按如下方式修改:
  % edate={December, 2005}
  %
  % 关键词用“英文逗号”分割
}

% 定义中英文摘要和关键字
\begin{cabstract}
  云存储的发展使得用户可以方便地存储、获取与分享数据。但与此同时,云存储也带来了很多安全问题,例如,数据隐私泄露等等。对称加密搜索的提出解决了数据隐私泄露问题,同时也保证了数据的可搜索性。通过使用对称加密搜索方案,用户可以在上传数据到云服务器之前,对数据进行加密,同时云服务器可以在用户的加密数据上进行搜索,从而确保数据隐私。

  然而,对称加密搜索的假定是云服务器是诚实且好奇的,即云服务器会遵守协议,但现实情况中云服务器往往是不可靠的。为了解决该问题,可验证对称加密搜索技术相应提出,它通过结果验证技术可以到检测云服务器的恶意行为。但是,现有的可验证对称加密搜索方案都不完善,例如,不支持用户数据动态更新,依赖于特定的对称加密搜索方案,只支持单用户读写等等。

  针对以上的问题,本文提出了一种通用的可验证对称加密搜索框架,该框架普适于任何加密搜索方案,支持用户数据更新,并且能够同时在单用户和多用户的场景下工作。本文的主要工作和创新点包括:

  \begin{itemize}
    \item 提出了一种单用户场景下的可验证对称加密搜索框架,并在此基础上设计了结果验证算法,该算法能同时保证数据新鲜性和数据完整性。该框架支持用户数据动态更新,并且将验证索引从对称加密搜索方案中解耦,使其可以为任何加密搜索方案提供结果验证功能。
    \item 提出了一种多用户场景下的可验证对称加密搜索框架。该框架支持单用户写,多用户读,确保了多用户场景下的数据新鲜性,并实现了数据共享场景下的结果验证。
    \item 本文采用了一个开源数据集作为测试数据,在本地环境对该框架进行了实验测试。安全性分析和实验表明,本文提出的可验证对称加密搜索框架不泄露数据隐私,并且给对称加密搜索方案引入的额外计算开销和通信开销很小,几乎可以忽略不计。
  \end{itemize}

\end{cabstract}

% 如果习惯关键字跟在摘要文字后面,可以用直接命令来设置,如下:
\ckeywords{对称加密搜索,结果验证,云存储}

\begin{eabstract}
  Cloud storage allows users to retrieve and share their data conveniently. %with well understood benefits, such as on-demand access, reduced data maintenance cost, and service elasticity.
  Meanwhile, cloud storage also brings serious data privacy issues, i.e., the disclosure of private information. In order to ensure data privacy without losing data usability,
  %a cryptographic notion named
  Searchable Symmetric Encryption (SSE) has been proposed. By using SSE, users can encrypt their data before uploading to cloud services, and cloud services can directly operate and search over encrypted data, which ensures data privacy.

  However, most SSE schemes only work with honest-but-curious cloud services that do not deviate from the prescribed protocols. This assumption does not always hold in practice due to the untrusted nature in storage outsourcing. To alleviate the issue, there have been studies on Verifiable Searchable Symmetric Encryption (VSSE), which functions against malicious cloud services by enabling results verification. But to our best knowledge, existing VSSE schemes exhibit very limited applicability, such as only supporting static database, demanding specific SSE constructions, or only working in the single-user model.

  In this paper, we proposed a generic verifiable SSE framework in both single-user model and single-writer multiple-reader model, which provides verifiability for any SSE schemes and further supports data updates. In summary, our contributions are three-fold:
  \begin{itemize}
    \item We proposed a verifiable SSE framework in single-user model and designed the result verification algorithms. The framework seperate the verification index from the SSE construction and can provides generic verification for any SSE schemes. The algorithms guaranteed both data freshness and data integrity with support of data updates.
    \item We proposed the first verifiable SSE framework in the single-writer and multiple-reader model, which ensures data freshness across multiple users and provides result verification under data sharing scenario.
    \item We implemented our framework in a local enviroment and fed it with an open-source data set. Rigorous analysis and experimental evaluations show that our shceme is secure and introduces small overhead for result verification.
  \end{itemize}
\end{eabstract}

\ekeywords{Searchable Symmetric Encryption, Result Verification,  Cloud Storage}
