\chapter{绪论}
\label{cha:intro}
\section{研究背景及选题意义}
云存储使得用户可以随时随地地存取数据,并且极大地方便了用户之间的数据共享,降低了维护数据的成本~\cite{juels2007pors,ateniese2008scalable,kamara2011cs2,wang2011enabling,stefanov2012iris,kamara2013parallel,sun2015catch}。但与此同时,云存储也带来了许多安全性问题,例如,数据丢失,数据隐私泄露等等。总体来说,云存储带来的安全性问题可以分为以下两类:
\begin{itemize}
	\item 可用性问题。要求云服务器保证数据不丢失,用户可以将云端作为数据中枢进行数据备份和同步。目前,一般的云服务提供商都采用了多副本的方式保障数据的可用性,即将数据的多个副本分别写入其他的存储节点,当一个节点发生故障时,其他节点上的数据继续提供服务,同时通过其他节点中的数据副本,快速恢复故障节点上丢失的数据。目前,针对数据可用性的相关学术研究包括数据拥有证明 (Proof of Data Possession, PDP)\cite{ateniese2007provable, ateniese2008scalable, erway2015dynamic,zhu2012cooperative} 以及数据可恢复性证明 (Proof of Retrievability, PoR)\cite{juels2007pors, bowers2009proofs, stefanov2012iris}。
	\item 隐私性问题。要求云服务器保证数据的隐私并且不泄露数据。目前,云服务提供商一般采用数据加密方式对隐私数据进行保护,但数据加密往往会导致数据可用性的降低,例如数据失去可搜索性。因此加密搜索 (Searchable Encryption, SE) 应运而生。加密搜索技术主要分为两类,一是对称加密搜索 (Searchable Symmetric Encryption, SSE)\cite{song2000practical,curtmola2011searchable,kamara2012dynamic,cash2014dynamic,wang2016searchable},二是公钥加密搜索 (Public Key Encryption with Keyword Search, PEKS)~\cite{boneh2004public}。
\end{itemize}
加密搜索的提出,使得用户可以在上传数据给云服务器之前,对其进行加密,并且使得云服务器可以在加密数据上进行搜索。从而既保证了数据隐私性,又保证了数据的可搜索性。目前,由于效率问题,应用较为广泛的为对称加密搜索技术。然而,大部分的对称加密搜索方案都基于服务器是诚实且好奇的假设~\cite{curtmola2011searchable, kamara2012dynamic, cash2014dynamic},即服务器会遵循协议但是可以从用户的查询中推断相关信息。这种假设在实际应用场景中往往是不成立的。因为云服务器可能会因为外部攻击,内部配置错误,软件错误等等问题而导致其违反原有协议~\cite{sun2015catch,bost2016verifiable}。这种协议违反所导致的最常见问题就是服务器返回的搜索结果不完整。例如,云服务器有可能为了节省计算开销和通信开销而返回少量搜索结果给用户,甚至有可能不返回搜索结果给用户。

为了解决该问题,可验证对称加密搜索技术也相应提出\cite{kamara2011cs2,kurosawa2012uc,chai2012verifiable,kurosawa2013update,stefanov2014practical,cheng2015verifiable,bost2016verifiable,ogataefficient}。可验证对称加密搜索技术允许用户对搜索结果进行验证,从而来检测服务器的不诚信行为,保障加密搜索的正确性。然而,据我们所知,现有的可验证对称加密搜索方案都是不完善的。例如,有的方案~\cite{kurosawa2012uc,chai2012verifiable,cheng2015verifiable,ogataefficient}不支持数据更新,只能作用在静态数据库中,数据库若有变化则需要重建整个索引。有的方案~\cite{kamara2011cs2,kurosawa2013update,stefanov2014practical}无法防止服务器故意返回空结果来规避结果验证。换句话说,以上这些方案\cite{kamara2011cs2,kurosawa2013update,stefanov2014practical}在用户提交的关键字不存在于数据库中时,是不返回任何搜索结果的,这就导致了服务器可以对任意关键字返回空结果来规避结果验证,除非用户在本地保留数据库的所有关键字集合。另外,大部分的可验证对称加密搜索方案~\cite{kamara2011cs2,kurosawa2012uc,chai2012verifiable,kurosawa2013update,stefanov2014practical,
cheng2015verifiable,ogataefficient,bost2016verifiable}仅仅支持在单用户场景下工作,即数据持有者自己写自己读的场景,而现实情况中,数据往往有共享需求,即一方写多方读的多用户场景\footnote{本文所述的多用户场景指一方写入,多方读取的场景,下文中若不做特别说明,均指这种情况。}。表格~\ref{tab:comparison}比较了现有的可验证对称加密搜索方案。


\begin{table*}[t]
  \begin{center}
  \caption{现有可验证对称加密搜索方案比较}
  \label{tab:comparison}
  %\begin{threeparttable}
  \begin{tabular}{c c c c c c}
    \hline
                                          &动态性$^1$         &新鲜性$^2$     &完整性$^3$    &验证效率$^4$        &通用性$^5$  \\
    \hline
    KPR11~\cite{kamara2011cs2}            &\checkmark          &\checkmark         &\texttimes                          &$O(|W|)$                      &\checkmark  \\

    KO12~\cite{kurosawa2012uc}            &\texttimes          &\text{-}           &\texttimes                          &$O(n)$                        &\texttimes\\

    CG12~\cite{chai2012verifiable}        &\texttimes          &\text{-}           &\checkmark                          &$O(log(|W|))$                 &\texttimes  \\

    KO13~\cite{kurosawa2013update}        &\checkmark          &\checkmark         &\texttimes                          &$O(n)$                        &\texttimes \\

    SPS14~\cite{stefanov2014practical}    &\checkmark         &\checkmark         &\texttimes                          &$min\{\alpha+log(N), r log^3(N)\}$                  &\texttimes \\

    CYGZR15\cite{cheng2015verifiable}    &\texttimes           &\text{-}         &\texttimes                            &$O(|W|)+O(r)$                 &\texttimes \\

    BFP16~\cite{bost2016verifiable}       &\checkmark          &\checkmark         &\checkmark                          &$O(r)$                        &\checkmark   \\

    OK16~\cite{ogataefficient}            &\texttimes          &\text{-}           &\checkmark                          &$O(r)$                        &\checkmark  \\

    我们的方案                             &\checkmark          &\checkmark         &\checkmark                          &$O(log(|W|))$                 &\checkmark  \\
    \hline
  \end{tabular}\\
  \end{center}
	$^1$ 注意,动态性是指方案是否支持用户数据动态更新,由此可将可验证对称加密搜索方案分为静态和动态两种类型,后者在功能性上更完善。\\
  $^2$ 注意,'\texttimes' 表示有实现的需求但是该方案没有实现, 而 '-' 表示没有实现的需求。具体而言,静态的可验证对称加密搜索方案不存在数据新鲜性问题,因此方案~\cite{kurosawa2012uc,chai2012verifiable,cheng2015verifiable,ogataefficient}也没有进行数据新鲜性验证的需求。\\
  $^3$ 我们考虑各种数据完整性攻击,尤其包括服务器故意返回空结果来规避结果验证的场景。\\
  $^4$ 验证效率是指服务器进行结果验证支持所需要的计算开销。对于表格中的非通用型方案~\cite{kurosawa2012uc,chai2012verifiable,kurosawa2013update,stefanov2014practical,cheng2015verifiable}来说,由于他们的方案并没有将验证索引从加密搜索方案中解耦,因此他们的验证效率和服务器进行加密搜索所需的计算开销是等价的。这里,$n$ 代表所有文件的数量, $|W|$ 表示所有关键字的数量, $r$ 表示包含某一特定关键字的文件数量, $\alpha$ 表示某一关键字历史上被加入到集合中的次数~\cite{stefanov2014practical}, $N$ 表示 (文件,关键字) 对的数量。\\
  $^5$ 一个通用的可验证对称加密搜索方案是指该方案可以为任何加密搜索方案提供结果验证,而非通用的可验证对称加密搜索方案表示该方案仅支持在特定的加密搜索方案下工作。\\
\end{table*}


\section{本文的主要内容}
本文基于默克尔帕特里夏树 (Merkle Patricia Tree, MPT) 和增量哈希 (Incremental Hash, IH) 技术,提出了一种单用户场景下的通用可验证对称加密搜索框架。该框架将验证索引从对称加密搜索方案中解耦,使其可以与任何对称加密搜索方案结合,包括但不限于论文~\cite{stefanov2014practical,cash2014dynamic,kamara2012dynamic}中的方案。该验证索引基于支持动态更新的数据结构MPT构建,因此支持用户动态更新其数据集,而不需要重新构建验证索引。该验证索引将加密后的关键字和其对应的文件存储于叶子节点中,从而使得MPT的根节点成为用户数据完整性的见证,用于后续支持结果验证。
同时,本文还提出了基于该验证索引的一系列验证机制,来确保数据完整性和数据新鲜性的验证。与以往的方案不同~\cite{kamara2011cs2,kurosawa2013update,stefanov2014practical},我们的方案要求服务器不管搜索关键字存在与否,都需要给用户返回一个“证明”,用于让用户验证服务器是故意返回了空结果还是搜索关键字的确不存在与现有数据集中。需要特别说明的是,我们的方案不需要用户在本地维护文件集对应的关键字集合。

此外,基于以上方案,本文利用时间戳链和公钥加密机制,首次提出了一种多用户场景下的通用可验证加密搜索框架。该框架通过时间戳链和公钥加密机制构建了出了与MPT根哈希相关的鉴别符,解决了多用户共享数据情况下的数据新鲜性验证问题,实现了多用户下的结果验证。

本文通过严格的安全性证明,确保了方案不泄露用户的数据隐私信息。另外,本文通过实验表明,单用户场景和多用户场景下的可验证加密搜索框架效率很高,与加密搜索法方案结合时,给加密搜索引入的额外开销很小,几乎可以忽略不计。


\section{本文的结构安排}
本文的结构如下,第~\ref{cha:intro} 章为绪论,介绍了研究背景、选题意义以及主要工作内容;第~\ref{cha:related} 章为相关研究综述,介绍了对称加密搜索、可验证对称加密搜索等相关工作的研究现状,并对本文用到的相关概念和先验知识进行了介绍;第~\ref{cha:single-user} 章为单用户下的可验证对称加密搜索方案研究,从适用场景、方案流程、算法分析、安全性证明和实验验证几个角度,完整的介绍了单用户场景下可验证对称加密搜索框架方案;第~\ref{cha:multi-user} 章为多用户下的可验证对称加密搜索方案研究,整体结构与第~\ref{cha:single-user} 章类似;第~\ref{cha:conclusion} 章总结了全文,并对可验证加密搜索领域未来可能的发展方向进行了分析。
