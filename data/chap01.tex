\chapter{绪论}
\label{cha:intro}
\section{加密搜索的研究背景}
云存储使得用户可以随时随地地存取数据,并且极大地方便了用户之间的数据共享。但与此同时,云存储带来了许多安全性问题,总体来说可以分为以下两类:
(1)可用性(availability)。要求云服务器保证数据不丢失,用户可以将云端作为数据中枢进行数据备份和同步。目前,一般的云服务提供商都采用了多副本的方式保障数据的可用性,即将数据的多个副本分别写入其他的存储节点,当一个节点发生故障时,其他节点上的数据继续提供服务,同时通过其他节点中的数据副本,快速恢复故障节点上丢失的数据。目前,针对数据可用性的相关学术研究包括数据拥有证明(Proof of Data Possession, PDP)以及数据可恢复性证明(Proof of Retrievability, PoR)。
(2)隐私性(privacy)。要求云服务器保证数据的隐私并且不泄露数据。目前,云服务提供商一般采用数据加密方式对隐私数据进行保护,但数据加密往往会导致数据可用性的降低,例如数据失去可搜索性,因此加密搜索(Searchable Encryption)应运而生。
加密搜索技术主要分为两类,一是对称加密搜索(Searchable Symmetric Encryption, SSE),二是非对称加密搜索(Searchable Asymmetric Encryption, SAE)。由于非对称加密搜索的效率问题,我们在这里主要关注对称加密搜索。

对称加密搜索的模型如图1所示。用户自行对数据进行加密并上传到云端,与此同时,用户还需额外上传一个加密索引(index)使得云可以通过该索引来搜索数据。当用户需要搜索数据时,生成一个陷门(trapdoor),该陷门与关键字相关,使得用户可以在不暴露关键字内容的情况下进行内容搜索。
加密搜索使得用户在保护数据隐私的同时,满足了其搜索需求,但加密搜索并不能保证搜索结果的正确性。也就是说,加密搜索的前提是云服务器是诚实的,即服务器会遵守与用户的协议来正确的执行搜索操作,然而实际应用中,云服务器往往是不可信的,例如,云服务器有可能为了节省计算开销和通信开销而返回少量搜索结果给用户,甚至有可能不返回搜索结果给用户。为了防止云服务器的不诚信行为,学术界又提出了可验证的对称加密搜索机制(Verifiable Searchable Symmetric Encryption, VSSE)。可验证的加密搜索允许用户对搜索结果进行验证,来检测服务器的不诚信行为,保障了加密搜索的正确性。

\section{加密搜索的选题意义}
在可验证加密搜索中,由于服务器不诚信导致的安全性攻击主要可以分为以下两种:
	重放攻击(Replay Attack):在加密搜索中,重放攻击是指服务器(攻击者)试图返回旧的搜索结果,而不是最新的搜索结果。我们用Δ_n={δ_1,δ_2,⋯,δ_n}来表示旧版本的数据集,用δ_(n+1)来表示最新的数据集,则服务器返回的搜索结果是数据集δ_i的搜索结果,其中1≤i≤n。
	数据完整性攻击(Data Integrity Attack):在加密搜索中,数据完整性攻击是指服务器(攻击者)试图不让用户获取完整的搜索结果。我们用τ来表示加密搜索中用户的搜索陷门,用户应该得到的搜索结果为F(τ),而服务器返回的搜索结果为G(τ),其中G(τ)⊂F(τ)并且G(τ)可能为∅。
重放攻击仅存在于动态的加密搜索方案中,在数据库静态的情况下不存在。但现实中,动态数据库较为常见,因此重放攻击是可验证加密搜索必须要解决的问题。数据完整性攻击不仅包括服务器少返回搜索结果的情况还包括了服务器不返回搜索结果来规避结果验证的情况。
现有的可验证对称加密搜索方案中[5]-[12],尚未有方案完善地解决了以上两种攻击,尤其是服务器通过返回空结果来规避结果验证的情况,并且现有的方案均未考虑数据共享,即数据用户并不只是数据拥有者一人的情况。基于以上几点,我们展开了可验证对称加密搜索的研究,并提出了一个多用户场景下的可验证加密搜索方案。

\section{本文的主要内容}

\section{本文的结构安排}
